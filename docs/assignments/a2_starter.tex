% This is a starter document for writing up your solutions to CSC236 W20's assignment 2
% See http://www.cs.toronto.edu/~colin/236/W20/assignments/ for more information on assignments, and helpful LaTeX resources.
% Use of this starter document is not required.
\documentclass[boldsans]{article}
%% Package imports
\usepackage{ccfonts} % Purely for aesthetics - can be removed
\usepackage{amsmath,amssymb}
\usepackage{hyperref}
\hypersetup{colorlinks=true} % Colored hyperlinks
\usepackage{changepage}   % for the adjustwidth environment

% Setting up code snippets
\usepackage{listings}
\usepackage{xcolor}
\definecolor{codegreen}{rgb}{0,0.6,0}
\definecolor{codegray}{rgb}{0.5,0.5,0.5}
\definecolor{codepurple}{rgb}{0.58,0,0.82}
\definecolor{backcolour}{rgb}{0.95,0.95,0.92}
 
\lstdefinestyle{mystyle}{
    commentstyle=\color{codegreen},
    keywordstyle=\color{magenta},
    numberstyle=\tiny\color{codegray},
    stringstyle=\color{codepurple},
    basicstyle=\ttfamily\small,
    breakatwhitespace=false,         
    breaklines=true,                 
    captionpos=b,                    
    keepspaces=true,                 
    numbers=left,                    
    numbersep=5pt,                  
    showspaces=false,                
    showstringspaces=false,
    showtabs=false,                  
    tabsize=2
}
\lstset{style=mystyle}

% Setting up environments for numbered lemmas/theorems/corollaries, and proofs
\usepackage{amsthm}
\newtheorem{theorem}{Theorem}[section]
\providecommand*{\theoremautorefname}{Theorem}
\newtheorem{lemma}[theorem]{Lemma}
\providecommand*{\lemmaautorefname}{Lemma}
\newtheorem{corollary}{Corollary}[theorem]
\providecommand*{\corollaryautorefname}{Corollary}

%%%%%%%%%%%%%%%%%%%%%%%%%%%%%%%%%%%%%%%%%%%%%%%%%%%%%%%%%%%%%%%%%%%%%%%%%%%%%%%%%%%
% Define a few convenient custom commands and aliases. Normally, these might go into a separate
% .sty file (e.g. helpers.sty), which we would then import with the command \usepackage{helpers}
% I'm including them directly in the main tex file in this case, just to make it simpler to share.

\newcommand{\N}{\mathbb{N}}
\newcommand{\NOT}{\neg}
\newcommand{\AND}{\wedge}
\newcommand{\OR}{\vee}
% Macros for proofs
\newcommand{\proofheader}[1]{\noindent \underline{#1}}
\newcommand{\base}{\proofheader{Base Case}:}
\newcommand{\istep}{\proofheader{Inductive Step}:}
\newenvironment{solution}
{\bigskip \noindent \textbf{Solution: \\}}
{}
% Inline comments for use in equation environments
\newcommand{\INLINE}[1]{\qquad \text{\# #1}}
\DeclareMathOperator{\len}{len}
\newcommand{\emptylist}{[~]}
%%%%%%%%%%%%%%%%%%%%%%%%%%%%%%%%%%%%%%%%%%%%%%%%%%%%%%%%%%%%%%%%%%%%%%%%%%%%%%%%%%%


\title{CSC236 Winter 2020 Assignment \#2}
% Replace the placeholder text on the below 2 lines with your name and utorid
\newcommand{\name}{Your full name goes here}
\newcommand{\utorid}{Your utorid}
\author{\name \\ \textit{\utorid}}

\begin{document}
\maketitle

\begin{enumerate}

% Q1
% (Feel free to delete the restatement of the questions, if you'd rather just go straight into your solution)
\item In lecture, we used the following recurrence to represent the steps taken by an implementation of mergesort on a list of size $n$:
\begin{equation*}
  T_0(n) = 
\begin{cases}
  1 &\text{if } n = 1 \\
  n + 2T_0(n/2) &\text{if } n > 1
\end{cases}
\end{equation*}
(This recurrence assumes $n$ is a power of 2, hence the absence of floor and ceiling. You may maintain this assumption throughout this question.)

In reality, some implementations of divide-and-conquer algorithms stop the recursion before the input size becomes trivial. For example, a programmer may find that their mergesort implementation ends up running a bit faster if they stop recursing when the list size is less than 10, sorting these small lists using selection sort.

Consider the following recurrence, which models this scenario:

\begin{equation*}
  T(n) = 
\begin{cases}
  c &\text{if } n \leq k \\
  n + 2T(n/2) &\text{if } n > k
\end{cases}
\end{equation*}

$k, c \in \N^+$ are fixed constants, where $k$ represents the largest problem size which is solved non-recursively, and $c$ represents the cost of solving these small problems.

\begin{enumerate}
\item Use unwinding\footnote{
\textbf{Logistical note:} If you wish to use tree diagrams for the unwinding portions of this question (parts (a) and (c)), you are welcome to include scanned hand-drawn images, or diagrams generated using other software. See \href{https://en.wikibooks.org/wiki/LaTeX/Importing_Graphics}{this chapter of the \LaTeX Wikibook} for information on including images in \LaTeX documents. You may also describe the solution tree without explicitly drawing it (a table may be helpful).
}
to find a closed form for $T(n)$ when $n \geq k$. (You do not need to prove that your closed form is correct, but it should be clear how you arrived at it.)

\begin{solution}
Your solution goes here.
\end{solution}

\item What is the big-$\Theta$ complexity of $T(n)$? Does it depend on $k$? Briefly justify your answer (no proof required). You may not assume $n \geq k$ for this part.

\begin{solution}
Your solution goes here.
\end{solution}

\item Rather than assigning a fixed cost to the $n \leq k$ case, it may be more realistic to use a function of $n$, since there are a range of input sizes which are handled non-recursively. Since selection sort is a $\Theta(n^2)$ algorithm, we'll define our new recurrence $T'(n)$ to be
\begin{equation*}
  T'(n) = 
\begin{cases}
  n^2 &\text{if } n \leq k \\
  n + 2T'(n/2) &\text{if } n > k
\end{cases}
\end{equation*}
Find a closed form for $T'(n)$ for $n \geq k$, and show how you got there. Rather than unwinding from scratch, you may find it simpler to build on your work from part (a).

\begin{solution}
Your solution goes here.
\end{solution}

\item Is $T'(n) \in \Theta(T(n))$? Why or why not? Briefly justify your answer. As in part (b), you may not assume $n \geq k$.

\begin{solution}
Your solution goes here.
\end{solution}

\end{enumerate}

% Q2: recursive correctness
\newpage
\item Our boss has tasked us with writing a program to find the \textit{unique} maximum of a non-empty list of positive integers. If there is no unique maximum, our program should signal this by returning a negative number. For example, on input \texttt{[5, 2, 1, 2]}, our algorithm should return 5. Given \texttt{[2, 1, 2]}, we may return -1, -2, -236, or any other negative number.

Below is our first attempt to solve this problem.\footnote{
You can download the code for this question from \url{http://www.cs.toronto.edu/~colin/236/W20/assignments/umax.py}
}

\begin{lstlisting}[language=Python]
def umax(A):
    if len(A) == 1:
        return A[0]
    head = A[0]
    tail = A[1:]
    tmax = umax(tail)
    if head == tmax:
        return -1
    elif head > tmax:
        return head
    else:
        return tmax
\end{lstlisting}

\begin{enumerate}
    \item Based on the informal specification above, write precise pre- and post-conditions for umax. Your postcondition should use symbolic notation rather than restating the English description above (``find the unique maximum...''). The following postcondition was used in lecture for the function \texttt{max}, which found the (not necessarily unique) maximum of a list. It may be a useful starting point:
    
    \medskip
    
    \texttt{max(A) = x} where $\left(\exists j \in \N, A[j] = x\right)$ $\AND$ $\left(\forall i \in \N, i < \mathrm{len}(A) \implies A[i] \leq x\right)$
    
    \medskip
    
    You may find it helpful to formally define `helper' functions or predicates, as is done in question 3.

\begin{solution}
Your solution goes here.
\end{solution}

\item The given Python code above has a bug. Demonstrate the bug by finding a value of $A$ which meets the precondition, where umax misbehaves. For the value of $A$ that you find, you should state the expected behaviour (according to your postcondition) and how it differs from the function's actual behaviour on that input.

\begin{solution}
Your solution goes here.
\end{solution}

\item Consider our second draft of the function umax below:
\begin{lstlisting}[language=Python]
def umax(A):
    if len(A) == 1:
        return A[0]
    head = A[0]
    tail = A[1:]
    tmax = umax(tail)
    if head == tmax:
        return -1 * head
    elif head > abs(tmax):
        return head
    else:
        return tmax
\end{lstlisting}
Prove that this function is correct with respect to the specifications you devised in part (a). 

\begin{solution}
Your solution goes here.
\end{solution}

\end{enumerate}

\newpage
\item The function \texttt{maj} takes as input a list of natural numbers and, if the list has a majority element (one that appears more often than all other elements combined), it returns that element. For example \texttt{maj([1, 3, 3, 2, 3])} returns 3.\footnote{
You can download the code for this question from \url{http://www.cs.toronto.edu/~colin/236/W20/assignments/maj.py}
}
\begin{lstlisting}[language=Python]
def R(A):
    B = []
    i = 0
    while i < len(A):
        a = A[i]
        b = A[(i+1) % len(A)]
        if a == b:
            B.append(a)
        i += 1
    return B

def maj(A):
    prev = A
    curr = R(A)
    while len(curr) != len(prev):
        prev = curr
        curr = R(curr)
    return curr[0]
\end{lstlisting}

Prove that maj is correct.

% Feel free to add any additional definitions to this section if you find them useful to your proof.
\section{Definitions}

% Macro for an indented informal English description of a definition
\newcommand{\gloss}[1]{
\begin{adjustwidth}{10mm}{}
\textit{ i.e. #1 }
\end{adjustwidth}
}

% We could write e.g. $COUNT(A, x)$, but the problem with this is that LaTeX will treat this as
% the product of variables C, O, U, N, and T, rather than a function name. These macros will
% make sure our custom functions are typeset properly. They can only be used in math mode.
\newcommand{\COUNT}{\operatorname{COUNT}}
\newcommand{\ad}{\operatorname{ADVANTAGE}}
\newcommand{\maj}{\operatorname{MAJORITY}}
\newcommand{\pairs}{\operatorname{PAIRS}}
\newcommand{\SUCC}{\operatorname{SUCC}}
\newcommand{\R}{\operatorname{R}}

We will use $\N^*$ to denote the set of lists of natural numbers. Each of the functions below takes as its first argument a list $A \in \N^*$.

$\COUNT(A, x)$: $| \{ i \in \N \mid A[i] = x\} |$

\gloss{the number of occurrences of $x$ in list $A$}

$\SUCC(A, i): \begin{cases}
  0 &\text{if } i = \len(A)-1 \\
  i+1 &\text{if } i < \len(A)-1
\end{cases}$

\gloss{the `next' index in list $A$ after index $i$, treating $A$ as a circular list (with the last index `wrapping around' back to index 0). Not defined if $i \geq \len(A)$.}

$\pairs(A, x)$: $|  \{ i \in \N \mid A[i] = A[\SUCC(A, i)] = x\} |$

\gloss{the number of $[x, x]$ pairs in list $A$, again treating $A$ as circular}

$\ad(A, x)$: $\COUNT(A, x) - len(A) / 2$

$\maj(A, x)$: $\ad(A, x) > 0$

\gloss{$x$ is the majority element of $A$}

% Since this proof will be somewhat long, organizing it into sections may be helpful.
% one idea is to have a section for proofs about the R helper function, and another
% for proofs about the behaviour of R. Alternatively, you might want to have one section
% for every 'major' theorem (see the appendix in the handout, which devotes a section 
% to the "R counts are pair counts" theorem).
\section{\texttt{R} proofs}

% You may treat this theorem as given (since it was proven in the appendix of the assignment handout).
% For examples of use of the amsthm environments like theorem, lemma, and proof, as well as the use of
% \ref{} to refer to theorems and lemmas, see the latex source for the appendix at:
% http://www.cs.toronto.edu/~colin/236/W20/assignments/a2q3.tex
% See also: https://en.wikibooks.org/wiki/LaTeX/Theorems
\begin{theorem}[R ``correctness'']
\label{rcorr}
Given any $A \in \N^*$, $\R(A)$ terminates and returns a list of natural numbers $B$ such that 
$\forall x \in \N, \COUNT(B, x) = \pairs(A, x)$
\end{theorem}

\section{\texttt{maj} proofs}

\end{enumerate}
\end{document}
